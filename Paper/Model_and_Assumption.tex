\section{Model and Assumption}
We consider a network model where N sensor nodes are deployed in an area of interest. Nodes are localized and their positions are available at a base station. In this paper, we consider an event-driven model in which sensor nodes are roughly global synchronized to detect incoming events nearby and obtain corresponding sensing readings. Similar to recent centralized approaches for network fault detection, we assume the sensing readings are collected to the base station. To be generic, we also introduce our design conceptually independent of the type of event used.
\subsection{Assumption on Monotonicity}
Many recent studies indicate that the environment is a dominating factor that affects the sensing and communication characteristics in sensor networks. It is therefore unrealistic to assume a particular mathematical model that describes the relationship between the sensing reading attenuation and the distance a signal travels. In this work, we take a much weaker assumption that the readings can generally reflect the relative distance from the nodes to the event. In other words, normally the sensing readings monotonically change as the distance becomes further. Although this assumption can be violated locally with environment noise, the general trend holds.